\documentclass{article}
\usepackage[utf8]{inputenc}
\usepackage[a4paper, top=3cm, bottom=3cm, left=2cm, right=2cm]{geometry}
\usepackage{amsmath}
\usepackage{amsthm}
\usepackage{amssymb}
\usepackage{fancyhdr}
\usepackage{hyperref}
\usepackage{graphicx}
\usepackage{subfig}

\pagestyle{fancy}
\fancyhf{}
\lhead{200020120, 200050129, 200050157}
\rhead{CS765 Project Part-1}
\cfoot{\thepage}

\newcommand{\B}[1]{\textbf{#1}}
\newcommand{\I}[1]{\textit{#1}}

\title{\textbf{CS765 Project Part-1 \\ {\ Simulation of a P2P Cryptocurrency Network}}}
\author{Sanchit Jindal (200020120), Sarthak Mittal (200050129), Virendra Kabra (200050157)}
\date{Spring 2023}

\begin{document}
\begin{sloppypar}       % for overfull, etc.

    \maketitle
    \tableofcontents
    \thispagestyle{empty}

    \newpage
    % \setcounter{page}{1}

    \section{Questions}
        
    \noindent \B{2. What are the theoretical reasons of choosing an exponential distribution for generating transactions?}
    \begin{itemize}
        \item The Poisson distribution models events occurring with a constant mean rate ($\lambda$) and independent of the time since the last event. Transactions follow this memorylessness property. So, the inter-arrival time of generating transactions follows an exponential distribution (with mean inter-arrival time $\frac{1}{\lambda}$).
    \end{itemize}

    \noindent \B{4. Why is the mean of $d_{ij}$ inversely related to $c_{ij}$? Give justification for this choice.}
    \begin{itemize}
        \item Link speed ($c_{ij}$) measures the number of bits that any of the two nodes can push through their connection. Queuing delay ($d_{ij}$) is the time for which a message waits in a queue at the respective node. If the link speed is less, then a message takes longer to transmit, and longer the waiting time of other messages at that node. Hence the inverse proportionality.
    \end{itemize}

\end{sloppypar}
\end{document}